
%%%%%%%%%%%%%%%%%%%%%%%%%%%%%%%%%%%%%%%%%%%%%%%%%%%%%%%%%%%
%% Inicio de la introducción
%%%%%%%%%%%%%%%%%%%%%%%%%%%%%%%%%%%%%%%%%%%%%%%%%%%%%%%%%%%

\chapter{Introduction}
%%---------------------------------------------------------

Ray Tracing is one of the most advanced, elegant and precise ways to produce computer graphics images. Although the original idea has existed since as early as the 16th century, it's first computerized version didn't appear until the year 1968. During the last decade it has become increasingly popular thanks to GPU acceleration, and many libraries have emerged to aid in the development of applications that use this technique.

Each library better suits a different purpose for the final piece of software, be it real time graphics for videogames or other real-time applications, or pursuing a higher image quality for visual effects in movies, animations, complex scientific visualizations, etc.

This work aims to compare the most prevalent technologies used to build Ray Tracing applications for each purpose, Vulkan and OptiX, in order to benchmark their behaviour under different circumstances. We will implement a series of test cases which will vary the ammount of rendered and loaded geometry as well as image quality settings, and record the following data from each one:

\begin{itemize}
    \item[*] Acceleration Structure Building Time
    \item[*] Frame Render Time
    \item[*] Video Memory Consumption
\end{itemize}

% La introducción del TFM debe servir para que los miembros del tribunal que evalúan el trabajo puedan comprender el contexto en el que se realiza el mismo, y los objetivos que se plantean.
%
% Esta plantilla muestra una estructura básica de una posible memoria final de TFM. Esta estructura es recomendable pero puede ser modificada según el tipo de TFM.
%
% El esquema básico de una memoria final de TFM sería el siguiente:
% \begin{itemize}
% \item[•] Resumen en valenciano, español e inglés (máximo 2 páginas cada uno)
% \item[•] Tabla de contenidos
% \item[•] Introducción (con los objetivos del TFM)
% \item[•] Desarrollo
% \item[•] Resultados
% \item[•] Conclusiones
% \item[•] Bibliografía (publicaciones utilizadas en el estudio y desarrollo del trabajo)
% \item[•] Anexo (es opcional y se pueden incluir tantos como se requieran)
% \end{itemize}
%
% En cualquier caso, es el/la director/a del TFM quien indicará al estudiante la estructura de memoria final que mejor se ajuste al trabajo desarrollado.


%%---------------------------------------------------------
