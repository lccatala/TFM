
%%%%%%%%%%%%%%%%%%%%%%%%%%%%%%%%%%%%%%%%%%%%%%%%%%%%%%%%%%%
%% Inicio del resumen. 
%%%%%%%%%%%%%%%%%%%%%%%%%%%%%%%%%%%%%%%%%%%%%%%%%%%%%%%%%%%

\chapter*{Resum}

El Traçat de Raigs és una tècnica elegant i relativament moderna per dibuixar gràfics en 3D. L'algorisme subjacent existeix des de fa dècades, havent-se fet servir en indústries on el temps de dibuixat és menys important, com el cinema. Tot i això no ha pogut usar-se de forma prevalent en indústries que requereixin gràfics en temps real, com els videojocs, fins fa pocs anys, gràcies a ser accelerat mitjançant l'ús de GPUs. Actualment hi ha múltiples tecnologies que s'encarreguen de gran part d'aquesta acceleració per nosaltres. Cadascuna té una audiència objectiu específica, ja sigui gràfics en temps real o gràfics fotorealistes. Dues de les llibreries més importants en aquests dominis són Vulkan i Nvidia OptiX, respectivament.

En aquest treball explorarem el desenvolupament d'un traçador de raigs emprant cadascuna d'aquestes i analitzarem el seu rendiment en multitud d'ajustaments gràfics, escenes i configuracions de maquinari. Les mètriques en què ens centrarem són el temps de construcció de lestructura dacceleració, temps de dibuixat dun fotograma i ús de memòria gràfica.

L'objectiu d'aquest treball és conèixer les limitacions i les peculiaritats d'ambdues API en diferents situacions, cosa que permetrà seleccionar la tecnologia més adequada segons els nostres requisits de temps de processament, memòria i complexitat d'escenes.

%%--------------
\newpage
%%--------------

\chapter*{Resumen}

El Trazado de Rayos es una técnica elegante y relativamente moderna para dibujar gráficos en 3D. El algoritmo subyacente existe desde hace décadas, habiéndose ha usado en industrias donde el tiempo de dibujado es menos importante, como el cine. Sin embargo no ha podido usarse de forma prevalente en industrias que requieran gráficos en tiempo real, como los videojuegos, hasta hace pocos años, gracias a ser acelerado mediante el uso de GPUs. Actualmente existen múltiples tecnologías que se encargan de gran parte de dicha aceleración por nosotros. Cada una tiene una audiencia objetivo específica, ya sea gráficos en tiempo real o gráficos fotorrealistas. Dos de las librerías más importantes en dichos dominios son Vulkan y Nvidia OptiX, respectivamente.

En este trabajo exploraremos el desarrollo de un trazador de rayos empleando cada una de éstas y analizaremos su rendimiento en multitud de ajustes gráficos, escenas y configuraciones de hardware. Las métricas en que nos centraremos son el tiempo de construcción de la estructura de aceleración, tiempo de dibujado de un fotograma y uso de memoria gráfica.

El objetivo de este trabajo es conocer las limitaciones y peculiaridades de ambas APIs en distintas situaciones, lo que permitirá seleccionar la tecnología más adecuada según nuestros requisitos de tiempo de procesamiento, memoria y complejidad de escenas.

%%--------------
\newpage
%%--------------


\chapter*{Abstract}

Ray Tracing is an ellegant and relatively modern technique to draw 3D graphics. Even though the underlying algorithm has existed for decades now, it's only been in recent years that it has taken a prevalent role in industries like videogames or cinema, due to it being accelerated through the use of GPUs. Nowadays there are multiple technologies that handle most of this acceleration for us. Each with a specific target audience, be it real time graphics or aiming for a higher image quality. Two of the most prevalent libraries for each domain are Vulkan and Nvidia OptiX, respectively. 

In this work we will be exploring the development of a ray tracer with each of these and analyzing their performance across a slew of graphical settings, scenes and hardware configurations. The metrics we will be focusing on will be acceleration structure build time, frame rendering time and graphics memory usage.

This work's goal is to know the limitations and quirks of each of these APIs in different situations, which will allow future programmers to choose the technology better suited to their requirements for processing time, memory and scene complexity.
%%%%%%%%%%%%%%%%%%%%%%%%%%%%%%%%%%%%%%%%%%%%%%%%%%%%%%%%%%%
%% Final del resumen. 
%%%%%%%%%%%%%%%%%%%%%%%%%%%%%%%%%%%%%%%%%%%%%%%%%%%%%%%%%%%
